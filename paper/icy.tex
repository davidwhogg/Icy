% This project is part of the Icy project
% Copyright 2017 the authors.

\documentclass[modern]{aastex61}

% aastex parameters
%%\hypersetup{linkcolor=red,citecolor=green,filecolor=cyan,urlcolor=magenta}
\received{not yet; THIS IS A DRAFT}
%\revised{not yet}
%\accepted{not yet}
%\submitjournal{ApJ}
\shorttitle{causal approach to photometry}
\shortauthors{wang et al.}

% typography
\setlength{\parindent}{1.\baselineskip}
\newcommand{\acronym}[1]{{\small{#1}}}
\newcommand{\project}[1]{\textsl{#1}}
\newcommand{\Kepler}{\project{Kepler}}
\newcommand{\Ktwo}{\project{K2}}
\newcommand{\Euclid}{\project{Euclid}}
\newcommand{\WFIRST}{\project{\acronym{WFIRST}}}

% affiliations
\newcommand{\ccpp}{\affiliation{%
    Center for Cosmology and Particle Physics,
    Department of Physics,
    New York University}}
\newcommand{\flatiron}{\affiliation{%
    Flatiron Institute, a division of the Simons Foundation}}
\newcommand{\cds}{\affiliation{%
    Center for Data Science,
    New York University}}
\newcommand{\mpia}{\affiliation{%
    Max Planck Institute for Astronomie, Heidelberg}}
\newcommand{\mpiis}{\affiliation{%
    Max Planck Institute for Intelligent Systems, T\"ubigen}}

\begin{document}\sloppy\sloppypar\raggedbottom\frenchspacing % trust me

\title{A causal-independence approach to crowded-field photometry}

\author{Dun Wang}
\ccpp

\author{Dan Foreman-Mackey}
\flatiron

\author[0000-0003-2866-9403]{David W. Hogg}
\ccpp
\flatiron
\cds
\mpia

\author{Bernhard Sch\"olkopf}
\mpiis

\begin{abstract}\noindent % trust me
% Context
The \Kepler, \Ktwo, \Euclid, and \WFIRST\ Missions all produce time-domain imaging of
crowded fields, with space-hardware optical stability and pixel-level
astrometric registration, but imperfectly known point-spread function.
In every pixel of such imaging, variations with time have contributions
from spacecraft and stellar variability, the latter coming from multiple
overlapping stars.
% Aims
Separation of time variability into multiple sources can be cast as a
question in the area of causal inference.
% Methods
Here we apply the technique of Independent Components Analysis (\acronym{ICA})
to perform non-parametric source separation in crowded-field imaging from
the \Kepler\ Mission (and artificial data with similar properties).
\acronym{ICA} capitalizes on the expectation of having greater non-Gaussianity
in the independent sources than in their co-added combination.
% Results
We find that \acronym{ICA} does plausibly separate causally distinct spacecraft
and stellar variability signals in the data sets, with spacecraft signals contributing
to all pixels, and stellar signals contributing to small pixel subsets.
Even though the method is given no information about the spatial proximity of pixels,
the stellar variability components obtain spatial footprints that resemble the
\Kepler\ pixel-convolved point-spread function.
These results provide evidence that causal-inference approaches could have an
impact for astronomical data analysis.
\end{abstract}

\keywords{foo --- techniques: photometric -- bar}

\section{Introduction}

Why do we need to do crowded-field photometry?

What are the traditional approaches; they require a PSF.

What can we do with Kepler-like data?

The ideas behind ICA and similar methods.

\section{Assumptions and method}

List assumptions as an enumerate here.

\section{Experiments and results}

\section{Discussion}

What would we need to do to operate under more relaxed assumptions; what are
we getting wrong by making the assumptions we made?

\acknowledgements
It is a pleasure to thank\ldots

Grant numbers

Software and facilities.

\end{document}
